%%%%%%%%%%%%%%%%%%%%%%%%%%%%%%%%%%%%%%%%%
% "ModernCV" CV and Cover Letter
% LaTeX Template
% Version 1.11 (19/6/14)
%
% This template has been downloaded from:
% http://www.LaTeXTemplates.com
%
% Original author:
% Xavier Danaux (xdanaux@gmail.com)
%
% License:
% CC BY-NC-SA 3.0 (http://creativecommons.org/licenses/by-nc-sa/3.0/)
%
% Important note:
% This template requires the moderncv.cls and .sty files to be in the same 
% directory as this .tex file. These files provide the resume style and themes 
% used for structuring the document.
%
%%%%%%%%%%%%%%%%%%%%%%%%%%%%%%%%%%%%%%%%%

%----------------------------------------------------------------------------------------
%	PACKAGES AND OTHER DOCUMENT CONFIGURATIONS
%----------------------------------------------------------------------------------------

\documentclass[11pt,a4paper,sans]{moderncv} % Font sizes: 10, 11, or 12; paper sizes: a4paper, letterpaper, a5paper, legalpaper, executivepaper or landscape; font families: sans or roman
\usepackage{xcolor}
\moderncvstyle{banking} % CV theme - options include: 'casual' (default), 'classic', 'oldstyle' and 'banking'
\moderncvcolor{blue} % CV color - options include: 'blue' (default), 'orange', 'green', 'red', 'purple', 'grey' and 'black'

\usepackage{lipsum} % Used for inserting dummy 'Lorem ipsum' text into the template

\usepackage[scale=0.78]{geometry} % Reduce document margins
%\setlength{\hintscolumnwidth}{3cm} % Uncomment to change the width of the dates column
%\setlength{\makecvtitlenamewidth}{10cm} % For the 'classic' style, uncomment to adjust the width of the space allocated to your name

% added
%\usepackage{fancyhdr,url,xspace,hyperref}
\renewcommand{\footrulewidth}{0pt}
\pagestyle{fancy}

%----------------------------------------------------------------------------------------
%	NAME AND CONTACT INFORMATION SECTION
%----------------------------------------------------------------------------------------

\firstname{\textsc{Yu}} % Your first name
\familyname{\textsc{Chen}} % Your last name
% All information in this block is optional, comment out any lines you don't need
\title{Curriculum Vitae}
%\address{}
\mobile{(518) 423-5526}
%\phone{(000) 111 1112}
%\fax{(000) 111 1113}
\email{cheny39@rpi.edu}
%\homepage{www.sites.google.com/site/whatshugo}{Website: sites.google.com/site/whatshugo} % The first argument is the url for the clickable link, the second argument is the url displayed in the template - this allows special characters to be displayed such as the tilde in this example
\homepage{www.linkedin.com/in/whatshugo}{LinkedIn: linkedin.com/in/whatshugo}
%\homepage{www.github.com/hugochan}{GitHub:github.com/hugochan}

%\extrainfo{additional information}
%\photo[70pt][0.4pt]{pictures/picture} % The first bracket is the picture height, the second is the thickness of the frame around the picture (0pt for no frame)
%\quote{"A witty and playful quotation" - John Smith}

%----------------------------------------------------------------------------------------


\begin{document}

\makecvtitle % Print the CV title

%----------------------------------------------------------------------------------------
%	EDUCATION SECTION
%----------------------------------------------------------------------------------------

\section{Education}

\cventry{\textsc{Aug. 2015} - \textsc{May. 2020 (expected)}}{Ph.D in Computer Science}{Rensselaer Polytechnic Institute}{Troy, NY}{}{\textit{GPA: 3.89/4.0}}  % Arguments not required can be left empty

\cventry{\textsc{Sep. 2014} - \textsc{Dec. 2014}}{Exchange student in Computer \& Information Science}{The University of Michigan-Dearborn}{Dearborn, MI}{}{}  % Arguments not required can be left empty
%------------------------------------------------
\cventry{\textsc{Sep. 2011} - \textsc{Jul. 2015}}{B.Eng. in Telecommunications Engineering}{The University of Electronic Science and Technology of China}{Chengdu, China}{}{\textit{GPA: 3.98/4.0}}
%{\textit{GPA: 3.98/4.0, Rank: 5/463}}

%\section{Masters Thesis}

%\cvitem{Title}{\emph{Money Is The Root Of All Evil -- Or Is It?}}
%\cvitem{Supervisors}{Professor James Smith \& Associate Professor Jane Smith}
%\cvitem{Description}{This thesis explored the idea that money has been the cause of untold anguish and suffering in the world. I %found that it has, in fact, not.}





%----------------------------------------------------------------------------------------
%	RESEARCH EXPERIENCE SECTION
%----------------------------------------------------------------------------------------

\section{Research \& Work Experience}

%\cventry{\textsc{Sep. 2016} - \textsc{Jan. 2017}}{Advisor: Prof. \href{http://www.cs.rpi.edu/~zaki/}{\textcolor{blue}{Mohammed J. Zaki}}}{\textsc{K-Competitive Autoencoder for Text Analytics}}{Troy, NY}{}{
%Data mining of large-scale discrete data sets with a focus on 3D surface denoising.
%%\newline{}
%\begin{itemize}
%\item Applied data mining and machine learning algorithms on large-scale data sets.
%\item Collected and compared existing approaches of 3D surface denoising.
%\item Designed and implemented a voxel-based fast surface propagation method to remove non-isolated and sharp featured surface outliers.
%\item More information \href{http://www.cs.rpi.edu/~cheny39/file/Voxel-based\%20Fast\%20Surface\%20Propagation\%20Method\%20of\%20Non-isolated\%20and\%20Sharp\%20Featured\%20Surface\%20Outlier\%20Removal.pdf}{\textcolor{blue}{here}}.
%\end{itemize}}





\cventry{\textsc{May. 2017} - \textsc{Present}}{Advisor: Prof. \href{http://www.cs.rpi.edu/~zaki/}{\textcolor{blue}{Mohammed J. Zaki}}}{\textsc{Graduate Research Assistant, RPI}}{Troy, NY}{}{
}

\cventry{\textsc{May. 2018} - \textsc{Aug. 2018}}{Manager: Dr. \href{https://www.linkedin.com/in/lazaros-polymenakos-54983b6}{\textcolor{blue}{Lazaros Polymenakos}}}{\textsc{AI Research Intern, IBM Research}}{Yorktown Heights, NY}{}{
Designed and implemented a knowledge-based and task-oriented conversational recommendation system.
}


\cventry{\textsc{Aug. 2015} - \textsc{May. 2017}}{}{\textsc{Graduate Teaching Assistant, RPI}}{Troy, NY}{}{
%\begin{itemize}
%\item  Grading students' assignments.
%\item  Holding office hours to help students with learning.
%\end{itemize}
}

\cventry{\textsc{Mar. 2015} - \textsc{May. 2015}}{}{\textsc{Python Web Developer Intern at Microoh}}{Chengdu, China}{}{
Implemented a personalized learning management system for online education.
%\begin{itemize}
%\item  Responsible for the background computing of LPS 2.0 which was a personalized learning management system.
%\end{itemize}
}


\cventry{\textsc{Sep. 2014} - \textsc{Dec. 2014}}{Advisor: Prof. \href{http://umdearborn.edu/cecs/CIS/faculty/extended.php?p_id=50}{\textcolor{blue}{Jie Shen}}}{\textsc{Research Assistant in the Virtual Engineering Lab, UM-Dearborn}}{Dearborn, MI}{}{
Designed and implemented a fast voxel-based surface propagation method for outlier removal in laser measurement.
%Data mining of large-scale discrete data sets with a focus on 3D surface denoising.
%Proposed a fast voxel-based surface propagation method for outlier removal. See the \href{http://www.cadconferences.com/CAD16_284-289.html\#.WRzF7zOZOV6}{\textcolor{blue}{paper}} and \href{https://github.com/hugochan/denoiser}{\textcolor{blue}{code}}.
%\newline{}
%\begin{itemize}
%\item Applied data mining and machine learning algorithms on large-scale data sets.
%\item Collected and compared existing approaches of 3D surface denoising.
%\item Designed and implemented a fast voxel-based surface propagation method for outlier removal. More infomation is \href{http://www.cs.rpi.edu/~cheny39/file/Voxel-based\%20Fast\%20Surface\%20Propagation\%20Method\%20of\%20Non-isolated\%20and\%20Sharp\%20Featured\%20Surface\%20Outlier\%20Removal.pdf}{\textcolor{blue}{here}}.
%\end{itemize}}
}
%------------------------------------------------

\cventry{\textsc{May. 2014} - \textsc{Jul. 2014}}{Advisor: Prof. \href{http://scholar.google.com/citations?user=MXgWgmEAAAAJ}{\textcolor{blue}{Tao Zhou}}}{Research Assistant in the Web Sciences Center, UESTC}{Chengdu, China}{}{
%Empirical analysis and recommendation algorithms design in social network area.
Discovered interesting human behavior patterns with temporal dynamics in social networks. 
%More information \href{http://www.cs.rpi.edu/~cheny39/file/empirical\%20analysis\%20of\%20online\%20social\%20networks.pdf}{\textcolor{blue}{here}}.
%\newline{}
%\begin{itemize}
%\item Analyzed a mass of data provided by multiple social network websites.
%\item Found interesting patterns of human behaviors with temporal dynamics in social networks.
%\item Designed and implemented novel and effective recommendation algorithms.
%\item More information is \href{http://www.cs.rpi.edu/~cheny39/file/empirical\%20analysis\%20of\%20online\%20social\%20networks.pdf}{\textcolor{blue}{here}}.
%\end{itemize}
}

%------------------------------------------------

%\cventry{\textsc{Apr. 2013} - \textsc{Aug. 2013}}{}{PHP Developer in the Star Studio, UESTC}{Chengdu, China}{}{
%\begin{itemize}
%\item Updated campus forum system for student organization focusing on web development.
%\item Maintained existing web systems.
%\end{itemize}}


%----------------------------------------------------------------------------------------
%	GRANTS SECTION
%----------------------------------------------------------------------------------------
%
%\section{Grants}
%
%\cventry{\textsc{Apr. 2013} - \textsc{Dec. 2013}}{The University of Electronic Science and Technology of China}{Core Developer in the Yellow Ginkgo Innovation Fund}{Chengdu, China}{}{A GSM-based remote water quality monitoring system providing simple interactions between monitors and monitoring center. 
%%\newline{}
%\begin{itemize}
%\item One of the three core developers in this fund.
%\item Took charge of the embedded development.
%\item Designed and implemented the embedded part of the system.
%\item Designed and implemented communication protocols between monitors and monitoring center.
%\item More information \href{https://github.com/hugochan/remote_water_quality_monitoring}{\textcolor{blue}{here}}.
%%See the project on Github: \href{https://github.com/hugochan/remote_water_quality_monitoring}{\textcolor{blue}{github.com/hugochan/remote\_water\_quality\_monitoring}}.
%\end{itemize}}
%
%%----------------------------------------------------------------------------------------
%
%\cventry{\textsc{Sep.}2012 - \textsc{Nov.}2013}{The University of Electronic Science and Technology of China}{Director in the University Students Technology Innovation Fund}{Chengdu, China}{}{A wireless pen which could record its tracks and transmitted them to a computer application through Bluetooth.
%\begin{itemize}
%\item Built a group with six members.
%\item Designed and implemented a wireless pen system based on an ARM microcontroller.
%\item Wrote a GUI program running on a computer.
%\item More information  \href{https://github.com/hugochan/wirelessShow_lowerComputer}{\textcolor{blue}{here}}.
%%See the project on Github: \href{https://github.com/hugochan/wirelessShow_lowerComputer}{\textcolor{blue}{github.com/hugochan/wirelessShow\_lowerComputer}}.
%\end{itemize}}
%

%----------------------------------------------------------------------------------------
%	Projects SECTION
%----------------------------------------------------------------------------------------


\section{Projects}

\cventry{\textsc{May. 2017} - \textsc{Present}}{Rensselaer Polytechnic Institute}{Personalized Search and Recommendation for Health Empowerment}{Troy, NY}{Advisor: Prof. \href{http://www.cs.rpi.edu/~zaki/}{\textcolor{blue}{Mohammed J. Zaki}}}{
Designed and developed a novel deep learning-based Q\&A  system for personalized food search and recommendation. 
Check out our \href{https://foodkg.github.io}{\textcolor{blue}{website}}.
}

\cventry{\textsc{Jun. 2019} - \textsc{Sep. 2019}}{Rensselaer Polytechnic Institute}{Graph Learning for Graph Neural Networks}{Troy, NY}{Mentor: Dr. \href{https://sites.google.com/a/email.wm.edu/teddy-lfwu/home?authuser=0}{\textcolor{blue}{Lingfei Wu}}}{
Designed and developed a novel iterative deep graph learning method for graph neural networks.
}

\cventry{\textsc{Mar. 2019} - \textsc{May. 2019}}{Rensselaer Polytechnic Institute}{Natural Question Generation}{Troy, NY}{Mentor: Dr. \href{https://sites.google.com/a/email.wm.edu/teddy-lfwu/home?authuser=0}{\textcolor{blue}{Lingfei Wu}}}{
%Designed and developed a novel system for natural question generation via reinforcement learning based graph-to-sequence model.
Designed and developed a reinforcement learning based graph-to-sequence model for natural question generation.
}

\cventry{\textsc{Dec. 2018} - \textsc{Feb. 2019}}{Rensselaer Polytechnic Institute}{Conversational Machine Reading Comprehension}{Troy, NY}{Mentor: Dr. \href{https://sites.google.com/a/email.wm.edu/teddy-lfwu/home?authuser=0}{\textcolor{blue}{Lingfei Wu}}}{
Designed and developed a  system for conversational machine reading comprehension via graph neural networks.
}

\cventry{\textsc{Feb. 2017} - \textsc{Jul. 2017}}{Rensselaer Polytechnic Institute}{Comparative Text Analytics via Topic Modelling in Banking}{Troy, NY}{Advisor: Prof. \href{http://www.cs.rpi.edu/~zaki/}{\textcolor{blue}{Mohammed J. Zaki}}}{
Applied topic modelling approaches to predict bank failures using a large set of SEC filings.
}



\cventry{\textsc{Aug. 2016} - \textsc{Feb. 2017}}{Rensselaer Polytechnic Institute}{Text Analytics via Topic Modelling and Text Representation}{Troy, NY}{Advisor: Prof. \href{http://www.cs.rpi.edu/~zaki/}{\textcolor{blue}{Mohammed J. Zaki}}}{
Designed and developed a novel autoencoder-based system for text analytics via competitive learning.
}





\cventry{\textsc{Oct. 2016} - \textsc{Dec. 2016}}{Rensselaer Polytechnic Institute}{Unsupervised Cluster Labeling}{Troy, NY}{Advisor: Prof. \href{http://nlp.cs.rpi.edu/hengji.html}{\textcolor{blue}{Heng Ji}}}{Designed an unsupervised algorithm to automatically pick descriptive, human-readable labels for the clusters of entities by learning to predict hyper-hyponym relationships via word embeddings.
Check out the Github repo: \href{https://github.com/congruili/NLP-Cluster-Labeling-project}{\textcolor{blue}{github.com/congruili/NLP-Cluster-Labeling-project}}.
}

%----------------------------------------------------------------------------------------


\cventry{\textsc{Apr. 2016}}{Rensselaer Polytechnic Institute}{Evaluating Countries and Products in International Trade}{Troy, NY}{}{
Designed an evolutionary bipartite graph approach to evaluate which countries do better and which products are more valuable in international trade.
More information is provided \href{http://www.cs.rpi.edu/~cheny39/file/2016DataThon.pdf}{\textcolor{blue}{here}}.
}

%----------------------------------------------------------------------------------------


\cventry{\textsc{Mar. 2016} - \textsc{May. 2016}}{Rensselaer Polytechnic Institute}{Predicting whose Papers are Accepted the Most}{Troy, NY}{Advisor: Prof. \href{http://www.cs.rpi.edu/~zaki/}{\textcolor{blue}{Mohammed J. Zaki}}}{
Designed multi-layered graph mining techniques to rank research institutes by predicting the number of accepted papers in incoming top conferences.
Check out the Github repo: \href{https://github.com/hugochan/KDDCUP2016}{\textcolor{blue}{github.com/hugochan/KDDCUP2016}}.
}

%----------------------------------------------------------------------------------------


\cventry{\textsc{Mar. 2015} - \textsc{May. 2015}}{The University of Electronic Science and Technology of China}{Finding Email Correspondents in Online Social Networks.}{Chengdu, China}{}{
Designed an effective algorithm to help find email correspondents in online social networks by leveraging user profiles and network structures.
More information is provided \href{http://www.cs.rpi.edu/~cheny39/file/technical\%20details.pdf}{\textcolor{blue}{here}}.
}

%----------------------------------------------------------------------------------------




%----------------------------------------------------------------------------------------

%
%\cventry{\textsc{Oct. 2014}}{The University of Michigan-Dearborn}{rUDP}{Dearborn, MI}{}{
%Reliable end-to-end in-order data delivery service inside the application layer on top of the UDP which was encapsulated as TCP-like stuff. More information  \href{https://github.com/hugochan/rUDP}{\textcolor{blue}{here}}.
%%See the project on Github: \href{https://github.com/hugochan/rUDP}{\textcolor{blue}{github.com/hugochan/rUDP}}.
%}

%----------------------------------------------------------------------------------------

%\cventry{\textsc{Oct. 2013} - \textsc{Dec. 2013}}{The University of Electronic Science and Technology of China}{Data Analysis of SNS}{Chengdu, China}{}{An application which could login Renren (a Chinese version of Facebook) and obtain user data to: 1) draw user relationship graphs. 2) recommend public pages. More information  \href{https://github.com/hugochan/RenrenDataRepo}{\textcolor{blue}{here}}.
%%See the project on Github: \href{https://github.com/hugochan/RenrenDataRepo}{\textcolor{blue}{github.com/hugochan/RenrenDataRepo}}.
%}

%----------------------------------------------------------------------------------------

%\cventry{\textsc{Apr. 2013} - \textsc{Jun. 2013}}{The University of Electronic Science and Technology of China}{CourseNow}{Chengdu,  China}{}{An automatic course-registration application aimed at UESTC which could 1) save users' preference lists of courses. 2) login (with cookie) and register multiple courses automatically and concurrently.  More information  \href{https://github.com/hugochan/CourseNow}{\textcolor{blue}{here}}.
%%See the project on Github:  \href{https://github.com/hugochan/CourseNow}{\textcolor{blue}{github.com/hugochan/CourseNow}}.
%}


%----------------------------------------------------------------------------------------
%	Publications
%----------------------------------------------------------------------------------------


\section{Publications}

\subsection{Preprints}
{\footnotesize
\begin{enumerate}

\item  \textbf{Yu Chen}, Lingfei Wu and Mohammed J. Zaki, Iterative Deep Graph Learning for Graph Neural Networks.
\item  \textbf{Yu Chen}, Lingfei Wu and Mohammed J. Zaki, Reinforcement Learning Based Graph-to-Sequence Model for Natural Question Generation.
\item  \textbf{Yu Chen}, Lingfei Wu and Mohammed J. Zaki, GraphFlow: Exploiting Conversation Flow with Graph Neural Networks for Conversational Machine Comprehension. 

 \end{enumerate}
}

 
\subsection{Conference Publications}
{\footnotesize 
\begin{enumerate}

\item \textbf{Yu Chen}, Lingfei Wu and Mohammed J. Zaki. Natural Question Generation with Reinforcement Learning Based Graph-to-Sequence Model. \textbf{In NeurIPS 2019 workshop on Graph Representation Learning} (NeurIPS GRL 2019), Vancouver, BC, Canada, Dec 8-Dec 14, 2019. 
\item Steven Haussmann, \textbf{Yu Chen}, Oshani Seneviratne, Nidhi Rastogi, James Codella, Ching-Hua Chen, Deborah McGuinness, Mohammed J. Zaki. FoodKG Enabled Q\&A Application. \textbf{In Proceedings of the 18th International Semantic Web Conference} (ISWC 2019), Auckland, New Zealand, Oct. 26-Oct. 30, 2019.
\item Steven Haussmann, Oshani Seneviratne, \textbf{Yu Chen}, Yarden Ne’eman, James Codella, Ching-Hua Chen, Deborah L. McGuinness and Mohammed J. Zaki. FoodKG: A Semantics-Driven Knowledge Graph for Food Recommendation. \textbf{In Proceedings of the 18th International Semantic Web Conference} (ISWC 2019), Auckland, New Zealand, Oct. 26-Oct. 30, 2019. 
\item \textbf{Yu Chen}, Lingfei Wu and Mohammed J. Zaki. GraphFlow: Exploiting Conversation Flow with Graph Neural Networks for Conversational Machine Comprehension. \textbf{In ICML 2019 Workshop on Learning and Reasoning with Graph-Structured Representations} (ICML LRG 2019), Long Beach, CA, June 9-June 15, 2019. 
\item \textbf{Yu Chen}, Lingfei Wu and Mohammed J. Zaki. Bidirectional Attentive Memory Networks for Question Answering over Knowledge Bases. \textbf{In Proceedings of the 2019 Annual Conference of the North American Chapter of the Association for Computational Linguistics} (NAACL-HLT 2019), minneapolis, MN, June 2-June 7, 2019. Long Oral Paper.
\item \textbf{Yu Chen}, Rhaad M. Rabbani, Aparna Gupta and Mohammed J. Zaki. Comparative Text Analytics via Topic Modeling in Banking. \textbf{In Proceedings of the 2017 IEEE Symposium Series on Computational Intelligence} (IEEE SSCI 2017), Hawaii, USA, Nov 27-Dec 1, 2017.
\item \textbf{Yu Chen} and Mohammed J. Zaki. KATE: K-competitive Autoencoder for Text. \textbf{In Proceedings of the 23rd International Conference on Knowledge Discovery and Data Mining} (ACM SIGKDD 2017), Halifax, NS, Canada, August 13-17, 2017. Full Oral Paper. Acceptance rate=8.6\% (64 out of 748). 
\item \textbf{Yu Chen}, Hao Chen and Jie Shen. Fast Voxel-based Surface Propagation Method for Outlier Removal. \textbf{In Proceedings of the 13th International CAD Conference}, Vancouver, BC, Canada, June 27-29, 2016.
\end{enumerate}
}


\subsection{Journal Publications}

{\footnotesize
\begin{enumerate}
\item Hao Chen, \textbf{Yu Chen}, Xu Zhang, Baiyuan Li, Xiaoqiang Liu, Xuefei Shi and Jie Shen. A Fast Voxel-based Method for Outlier Removal in Laser Measurement. \textbf{In International Journal of Precision Engineering and Manufacturing}, 2019. 
\end{enumerate}
}

\section{Patents}
{\footnotesize
\begin{enumerate}



\item Lingfei Wu, \textbf{Yu Chen}, Mohammed J. Zaki. Method and System for Natural Question Generation via Reinforcement Learning Based Graph-to-Sequence Model. To be filed, 2019.
\item Lingfei Wu, \textbf{Yu Chen}, Mohammed J. Zaki. Conversation History Within Conversational Machine Reading Comprehension. Aug, 2019.

\end{enumerate}
}


%----------------------------------------------------------------------------------------
%	AWARDS SECTION
%----------------------------------------------------------------------------------------

\section{Honors \& Awards}

\cventry{\textsc{Jul. 2017}}{ACM SIGKDD}{Student Travel Award of SIGKDD 2017}{}{}{}  % Arguments not required can be left empty

\cventry{\textsc{Apr. 2016}}{Rensselaer Polytechnic Institute}{Second Place at the 2016 DataThon}{}{}{}  % Arguments not required can be left empty
%------------------------------------------------
\cventry{\textsc{2012 - 2013 \& 2013 - 2014}}{The University of Electronic Science and Technology of China}{The First-Class People's Scholarship}{}{}{}  % Arguments not required can be left empty
%------------------------------------------------
%\cventry{\textsc{Oct. 2012}}{China Association for Science and Technology \& ASUSTeK Computer Inc.}{Third Prize Outstanding Volunteer in Popular Science}{}{}{}  % Arguments not required can be left empty
%------------------------------------------------
\cventry{\textsc{2011 - 2012}}{Ministry of Education of China}{National Scholarship}{}{Top 1.6 \%}{}




%----------------------------------------------------------------------------------------
%	COMPUTER SKILLS SECTION
%----------------------------------------------------------------------------------------

\section{Skills}

\cvitem{Research}{Machine Learning, Deep Learning, Natural Language Processing, Data Mining}
\cvitem{Programming}{\textsc{Python} = \textsc{C/C++} > \textsc{Matlab} >  \textsc{R} = \textsc{Scheme} = \textsc{JavaScript} = \textsc{PHP}}
\cvitem{Software}{PyTorch, TensorFlow, Keras, Linux, MacOS, Database, Git}
%\cvitem{Others}{Data Mining, Machine Learning, Signal Processing, Web Development, Embedded Systems}

%----------------------------------------------------------------------------------------
%	VOLUNTEER EXPERIENCE SECTION
%----------------------------------------------------------------------------------------

%\section{Volunteer Experience}

%\cventry{\textsc{Aug.2012}}{China Association for Science and Technology \& Asustek Computer Inc.}{Group %Leader in the Asus College Science Volunteer Action}{Ningbo, China}{}{
%\begin{itemize}
%\item Popularized IT knowledge in a rural area.
%\item Taught children ages 7-13 basic computer skills.
%\end{itemize}}


%----------------------------------------------------------------------------------------
%	STANDARDIZE TESTS SECTION
%----------------------------------------------------------------------------------------

%\section{Standardized Test Scores}
%\cvitem{TOEFL}{28(Reading) + 25(Listening) + 17(Speaking) + 26(Writing) = 96}
%\cvitem{GRE}{154(Verbal) + 167(Quantitative) + 3.5(A.W.)}


%----------------------------------------------------------------------------------------
%	LANGUAGES SECTION
%----------------------------------------------------------------------------------------

%\section{Languages}

%\cvitemwithcomment{Chinese}{Mothertongue}{}
%\cvitemwithcomment{English}{Intermediate}{Conversationally fluent}


%----------------------------------------------------------------------------------------
%	COVER LETTER
%----------------------------------------------------------------------------------------

% To remove the cover letter, comment out this entire block

%\clearpage

%\recipient{HR Department}{Corporation\\123 Pleasant Lane\\12345 City, State} % Letter recipient
%\date{\today} % Letter date
%\opening{Dear Sir or Madam,} % Opening greeting
%\closing{Sincerely yours,} % Closing phrase
%\enclosure[Attached]{curriculum vit\ae{}} % List of enclosed documents

%\makelettertitle % Print letter title

%\lipsum[1-3] % Dummy text

%\makeletterclosing % Print letter signature

%----------------------------------------------------------------------------------------

\end{document}
